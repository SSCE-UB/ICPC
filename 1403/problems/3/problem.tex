\problemWithTime{Stars}{1 second}



People generally don’t care to give attention to stars in a moonlit night. In most cases the attention goes towards the moon. Sadly, you have to write a program now that can count the stars in the sky. For this problem a sky is a two-dimensional grid. Empty pixel is denoted by a ‘.’ (ASCII value 46) and a non-empty pixel is denoted by a ‘*’ (ASCII value 42). As a star is a very small object so it cannot occupy more than one pixel and in our sky two stars are never adjacent. So two or more adjacent non-empty pixels can denote some larger objects like moon, comet, sun or UFOs but they never represent a star. All the eight possible pixels around a pixel are adjacent to it. In the figure below the black pixel at the center has eight adjacent pixels. Of them three pixels are non-empty.

\vspace{.6cm}

\begin{center}
	\begin{verbatim}
		* . .
		. * *
		. . *
	\end{verbatim}
\end{center}

\mysec{Input}

The input file contains at most 1000 sets of inputs. The description of each set is given below:

\vspace{.2cm}


Each set starts with two integer numbers $r$ and $c$ ($0 < r, c < 101$), which indicates the row and column number of the image to follow. Next $r$ rows describe the sky as mentioned in the problem statement.

Input is terminated by a line containing two zeroes.

\mysec{Output}

For each set of input produce one line of output. This line contains a decimal integer which denotes the number of stars in the given sky.



\begin{tabular}{|>{\arraybackslash}m{9cm}|>{\arraybackslash}m{6cm}|}
	\hline
	Standard Input & Standard Output \\
	\hline
5 5  & 1  \\
. . . . .  & 3   \\
. . . . *  &   \\
. . . . *  &   \\
. . . * .  &   \\
* . . . .  &   \\
4 3  &   \\
. . .  &   \\
. * .  &   \\
. . .  &   \\
* . *  &   \\
0 0  &   \\
	\hline
\end{tabular}
