\problemWithTime{Grading System}{1 second}

Professor Hedayati has set specific rules for grading in his Operating Systems course based on student absences. The rules are as follows:

\begin{itemize}
	\item If a student has zero absences, they will receive a grade of 20.
	\item If a student has exactly seven absences, they will retain their current grade \( X \).
	\item Otherwise, for each day a student is absent (from 1 to 6 days or more than 7 days), their grade will decrease by exactly one point per day (if the grade drops below zero, it will be considered as zero).
\end{itemize}

Given the current grade \( X \) of a student and the number of absences \( N \), determine the student's final grade.

\mysec{Input}
\begin{itemize}
	\item The first line contains an integer \( X \) ($0 \leq X \leq 20$) which represents the student's current grade in the Operating Systems course.
	\item The second line contains an integer \( N \) ($ 0 \leq N \leq 100$) which represents the number of days the student was absent.
\end{itemize}


\mysec{Output}
Print the final grade the student will receive after considering the number of absences.


\vspace{1cm}

\begin{tabular}{|>{\arraybackslash}m{9cm}|>{\arraybackslash}m{6cm}|}
	\hline
	Standard Input & Standard Output \\
	\hline
	14  & 20 \\
	0  &  \\
	\hline
	6 & 6 \\
	7 &  \\
	\hline
\end{tabular}